\label{app:B}

\definecolor{mGreen}{rgb}{0,0.6,0}
\definecolor{mGray}{rgb}{0.5,0.5,0.5}
\definecolor{mPurple}{rgb}{0.58,0,0.82}
\definecolor{backgroundColour}{rgb}{0.95,0.95,0.92}

\lstdefinestyle{CStyle}{
	%backgroundcolor=\color{backgroundColour},   
	commentstyle=\color{mGreen},
	keywordstyle=\color{blue},
	numberstyle=\tiny\color{mGray},
	stringstyle=\color{magenta},
	basicstyle=\scriptsize,
	breakatwhitespace=false,         
	breaklines=true,                 
	captionpos=b,                    
	keepspaces=true,                 
	%numbers=left,                    
	numbersep=5pt,                  
	showspaces=false,                
	showstringspaces=false,
	showtabs=false,                  
	tabsize=2,
	language=Python
}
\fontfamily{ptm}\selectfont
This appendix contains the script we created to measure the performance overhead of \codeft{ferify}.
\fontfamily{qcr}\selectfont
\begin{lstlisting}[style=CStyle]
#!/usr/bin/python3

import time
from timeit import default_timer as timer
import math
import os
import sys

def test(iterations = 100, sleep_time = 1):
	timing_overhead = []
	total_timing = 0

	o_results = []
	o_total = 0
	o_avg = 0
	o_s = 0
	o_var = 0

	u_results = []
	u_total = 0
	u_avg = 0
	u_s = 0
	u_var = 0

	r_results = []
	r_total = 0
	r_avg = 0
	r_s = 0
	r_var = 0

	for i in range(iterations):
		f = open("test" + str(i) + ".txt", "w")
		f.close()

	o_start = timer()
	for i in range(iterations):
		time.sleep(sleep_time)
	o_stop = timer()
	baseline = o_stop - o_start

	f = [None] * iterations
	o_start = timer()
	for i in range(iterations):
		f[i] = open("test" + str(i) + ".txt", "w")
		time.sleep(sleep_time)
	o_stop = timer()
	o_overhead = ((o_stop - o_start) - baseline) / iterations

	for i in range(iterations):
		f[i].close()

	r_start = timer()
	for i in range(iterations):
		os.rename("test" + str(i) + ".txt", "test" + str(i) + "_.txt")
		time.sleep(sleep_time)
	r_stop = timer()
	r_overhead = ((r_stop - r_start) - baseline) / iterations

	u_start = timer()
	for i in range(iterations):
		os.unlink("test" + str(i) + "_.txt")
		time.sleep(sleep_time)
	u_stop = timer()
	u_overhead = ((u_stop - u_start) - baseline) / iterations

	print("Timing:\t" + str(baseline))
	print("Open:\t" + str(o_overhead))
	print("Rename:\t" + str(r_overhead))
	print("Unlink:\t" + str(u_overhead))


if __name__ == '__main__':
	if len(sys.argv) == 3:
		test(int(sys.argv[1]), int(sys.argv[2]))
	else:
		test()

\end{lstlisting}

\fontfamily{ptm}\selectfont

%\lipsum*[65]
%
%When an equation occurs in the middle of a sentence, such as this one involving $e \in \mathbb{R}$,
%\begin{eqnarray}
% e^x &\approx& 1+x+x^2/2! \nonumber \\
%   && \hphantom{1} + x^3/3! + x^4/4! \nonumber \\
%   && \hphantom{1} + x^5/5!,
%\end{eqnarray}
%then we need proper punctuation (such as the comma above) and the sentence ends here, on the next line.
%
%\section{Section Example}
%\lipsum[47]
%
%\begin{figure}[!htb]
%\framebox[\textwidth]{\parbox{\textwidth}{\lipsum[65]}}
%\caption{Short figure title, with \emph{emph} and \textit{italics} in a caption.}
%\caption*{\small This is the long caption that explains the figure in detail and
%expounds on its relevance to the text.
%All figures need to be referenced in the text before the image or table.
%Full source citation, as applicable, is required.
%%Source~\cite{IEEEexample:bibtexguide}: \bibentry{IEEEexample:bibtexguide}
%}
%\end{figure}
%
%\subsection{Subsection Example}
%\lipsum[56]
%
%\section{Another Section}
%\lipsum[55-56]
%
%\begin{figure}[!htb]
%\framebox[\textwidth]{\parbox{\textwidth}{\lipsum[65]}}
%\caption{Some styled math in a caption, $\mathsf{Func}(x, \sigma) = x^2 + \overline{\sigma} + \pi$.}
%\caption*{\small This is the long caption that explains the figure in detail
%and expounds on its relevance to the text.
%This figure is original and requires no citation.}
%\end{figure}
%
%\begin{figure}[!htb]
%\centering
%\subfloat[First sub-figure]{
%   \framebox[0.47\textwidth]{\parbox{0.45\textwidth}{\lipsum[65]}}
%}
%\hfill
%\subfloat[Second sub-figure]{
%   \framebox[0.47\textwidth]{\parbox{0.45\textwidth}{\lipsum[65]}}
%}
%\caption{Caption using subfigure package.}
%\caption*{\small This is the long caption that explains the figure in detail
%and expounds on its relevance to the text.
%This figure is original and requires no citation.}
%\end{figure}
%
%
