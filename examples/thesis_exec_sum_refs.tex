% This starts the document; the "npsreport" style is really a modified
% report style. Feel free to use the options explained in the technical
% report NPS-CS-11-011, included under the doc/ directory.
\documentclass[twoside,thesis]{npsreport}

%
% Put extra packages you may need to customize your thesis
%
\usepackage{doc,lipsum} % provides \BibTex and \lipsum macros, for demos

%
% For Example: you might find one of these useful:

%\usepackage{epstopdf}        % to use .eps files for figures
%\usepackage{bm}              % bold math if you need bold greek letters
%\usepackage{glossaries}      % see http://en.wikibooks.org/wiki/LaTeX/Glossary
%\usepackage{asymptote}       % for graphics
% The asymptote package allows for very nice graphics and figures
% Proper usage requires additional information located at:
% http://asymptote.sourceforge.net/
% See the gallery at this URL for examples

%\usepackage{placeins}        % float placement
% Provides \FloatBarrier which keeps figures/tables in the same section.
% LaTeX sometimes moves them to fill up pages.
% http://ftp.math.purdue.edu/mirrors/ctan.org/macros/latex/contrib/placeins/placeins-doc.pdf

%\usepackage[numbered]{mcode} % matlab code
% The mcode package must be separately downloaded.
% http://www.mathworks.com/matlabcentral/fileexchange/8015-m-code-latex-package

%\usepackage{flafter}         % float placement
% Ensures that figures/tables do not appear in the document before
% they are referenced in the text.

% This package lets us build references that appear after the Executive Summary
\usepackage{bibunits}


\title{[Title]}

% Student info
\author{[Author Name]}
\rank{[Rank, Service]}    %\rank{Civilian} % if you don't have a rank
\degree{Master of Science in [Degree]}
\degreeabbreviation{MS}   % Should be MS, MBA or MA
\prevdegrees{[B.S., My Old School, Year]} % previous degree

% Department info
\department{Department of [Department]}
\thesisadvisor{[Primary Advisor]}
\secondreader{[Second Reader]}
\departmentchair{[Department Chair]}

% The date you are graduating:
\degreedate{[Month Year]}

% See Thesis processor's release form for approved distribution statements.
\distribution{Approved for public release. Distribution is unlimited}

% Your abstract.  New paragraphs start after an empty line.
\abstract{%
\lipsum[1] % example text, remove me
}

% Switch the below lines around, if FOUO
\securitybanner{}  %\securitybanner{FOR OFFICIAL USE ONLY}

%
% Mandatory fields for the SF298.
%
\ReportType{Master's Thesis}
\ReportDate{[Month Year]}       % for a thesis, graduation date
\SponsoringAgency{N/A}          % really, for technical reports
\DatesCovered{MM-DD-YYYY to MM-DD-YYYY}
\ReportClassification{Unclassified}
\AbstractClassification{Unclassified}
\PageClassification{Unclassified}
%
% Optional fields for the SF298.
%
\RPTpreparedFor{}
\ContractNumber{}
\GrantNumber{}
\ProgramElementNumber{}
\TaskNumber{}
\WorkUnitNumber{}
\POReportNumber{}
\Acronyms{}
\SMReportNumber{}
\SubjectTerms{}
\ResponsiblePerson{}
\RPTelephone{}
\SignatureOne{}
\SignatureTwo{}
\SupplementaryNotes{The views expressed in this document are those of
  the author and do not reflect the official policy or position of the
  Department of Defense or the U.S. Government. %
  IRB Protocol Number: N/A. % if you need to note an IRB Protocol or N/A
}

% Optional. Prevents footnotes from being reset at each chapter
% Comment this out to have them reset with each chapter.
\makeatletter
\@removefromreset{footnote}{chapter}
\makeatother

% Optional. Adds pdf metadata and links.
% This should be right before the \begin{document}, to be the
% last package / macros defined. (Hyper-ref is fragile,
% needs to be last, and has known conflicts with other packages.)
% Comment out if you have build problems building with hyperref
\NPShyperref

%
% Your thesis begins here
%
\begin{document}

\NPScover                  % Cover page
\NPSsftne                  % SF298 form
%\NPSsignature             % Tech Report page (iii): signature page
\NPSthesistitle            % Thesis page (iii): title page
\NPSabstractpage           % Abstract Page
\NPSfrontmatter            % NPS front matter follows

% This changes the chaptermark and includes the various tables
% It must be here.
\renewcommand{\chaptermark}[1]{\markboth{\MakeUppercase{\chaptername}\ \thechapter.\ #1}{}}

%
% If you don't need one of these, comment it out.
%
\NPStableOfContents
\NPSlistOfFigures
\NPSlistOfTables
\NPSlistOfAcronymsFromFile{acronyms}

%
% Put Executive summary here.
% New paragraphs start after an empty line.
%
\NPSexecsummary{
\begin{bibunit}[nps_thesis] % START: bibunit, use nps_thesis.bst for the style

This is an example of how to create an executive summary with its own references 
section using the \texttt{bibunit} package.
The build process needs to change to accomodate this. 
The \texttt{bibunit} package builds separate unit files
\texttt{bu1.aux}, \texttt{bu2.aux}, etc. 
These needs to be run through Bib\TeX{} separately.
In this example, the executive summary if the first (and only) bibunit, 
so we need to do the commands:
\begin{itemize}
	\item[] \texttt{pdflatex report}
	\item[] \texttt{bibtex report}
	\item[] \texttt{bibtex bu1}
	\item[] \texttt{pdflatex report}
	\item[] \texttt{pdflatex report}
\end{itemize}
The \texttt{Makefile} demonstrates how to script this.

\section*{Executive Summary Section}
The references~\cite{IEEEexample:incollectionmanyauthors},
\cite{IEEEexample:articledualmonths}
are in both the summary,
and in the final references; they are numbered separately.
Some references~\cite{IEEEexample:shellCTANpage,IEEEexample:bibtexuser,
IEEEexample:article_typical} only appear in this section,
and are not part of the final bibliography.

Note, you cannot use numbered sections in the executive summary,
since the summary has no number itself.

This sentence demonstrates that acronyms, like \ac{US}, work in the
executive summary; the \ac{US} is the short version. The counter will be re-set
in the main body, where its first use will be long again.

\subsection*{An Exec Summary Subsection}
\lipsum[2-3] % example text; remove me

%
% this makes references appear at the section-level instead of chapter-level
%
\begingroup
 \let\stdthebibliography\thebibliography
 \renewcommand{\thebibliography}{%
 \let\chapter\subsubsection
 \titlespacing*{\subsubsection}{0pt}{5ex plus 2ex}{-1ex plus .2ex}
 \stdthebibliography}
 \raggedright     % don't automatically full justify bibliographic references
 \singlespacing   % reduce extra line breaks between entries
 \small
 \putbib[thesis]  % use thesis.bib and place the bibliography here
\endgroup
%
% This is the end of special macros that tweak the appearance of the references
%
\end{bibunit}    % END: bibunit
}

%
% Put acknowledgements here.  
% New paragraphs start after an empty line.
%
\NPSacknowledgements{%
\lipsum[1-3] % example text; remove me
}

% Start layout for the NPS body
\NPSbody


% CHAPTERS
% You have two options on how to structure your thesis:
% a) A single file. All chapters, sections, etc. go in this file.
%    This can make navigating your thesis a little more difficult.
% b) Use multiple files.  One chapter per file is recommended.
%    This breaks your thesis up into logical units to edit.
%
\chapter{Introduction}\label{ch:intro}

System virtualization, which has been increasing in popularity over the last few years, makes it possible to run multiple different \acp{OS} on the same physical machine. \acp{VM} are run independently of each other on the same physical machine, known as a host, without any indication that there is another \ac{OS} running on the same host. The software that facilitates this resource sharing capability is called a hypervisor.

\par The emergence of cloud computing has increased the need for many new and different services from global vendors. According to the \ac{NIST}~\cite{mell2011nist}, "Cloud computing is a model for enabling ubiquitous, convenient, on-demand network access to a shared pool of configurable computing resources (e.g., networks, servers, storage, applications, and services) that can be rapidly provisioned and released with minimal management effort or service provider interaction."  

\par Virtualization solved the increasing requirement for resources for these services to run on. Instead of having many separate physical machines running the required various software, which usually results in under-utilization, one machine with better technical specifications and capabilities\footnote{such as more memory capacity and multi-core \acp{CPU}} was used; with virtualization, each vendor could run services on a dedicated \ac{VM}. 

\par In order to improve network security on these machines, as well as redundancy among different services service providers started using many different \acp{VM} per vendor, instead of having one \ac{VM} running all the required services. In addition to requiring fewer resources by running one or two services, vendors could be reassured by knowing that if one VM fails, the rest of the services keep running. Furthermore, having each \ac{VM} run only a few services significantly reduces the attack surface available for possible vulnerability exploitation.

\par This increase in the use of virtualization has driven hardware manufacturers, like Intel and AMD, to introduce special virtualization \ac{CPU} instructions that demonstrate better, more reliable, and more secure allocation, sharing, usage, and performance.

\section{Problem Statement}\label{sec:problem} 
When an \ac{OS} runs directly on a physical machine, it allocates and uses its resources to protect itself from network or other types of attacks. When it runs on a virtualization platform however, the hypervisor stands between the hardware and the running software, and can monitor what is happening inside a \ac{VM}. 

\par Despite the evolution of \ac{CPU} virtualization instructions and the continuous development of more efficient and secure hypervisors, the bottom-line remains the same: a \ac{VM} is still a system with all the vulnerabilities of its running \ac{OS} and software. At some point in time, it will be the victim of a successful exploitation. 

\par Although simple to manage and efficient, the native Linux file permission system lacks fine-grained user/group access to files. Once users belong to a group, nothing prohibits them from accessing all the files accessible to that group. Furthermore, when attackers gain access to a system, they will usually try to escalate their privileges by having access to the root account. After privilege escalation there is unrestricted access to the entire system and nothing out of reach; the attackers are free to read and modify files and change the system's configuration to their liking in order to serve their purposes.

\par Garfinkel et al~\cite{garfinkel2003virtual} introduced a new technique which leverages the hypervisor's viewing ability. \ac{VMI} is the “approach of inspecting a virtual machine from the outside for the purpose of analyzing the software running inside it.” In this context, outside means that the inspecting application resides outside the monitored \ac{VM} and can access the \ac{VM}'s state through the hypervisor. Because a system will be eventually subverted, we wanted to leverage the introspection capability of a hypervisor to try to protect critical files for the OS, the user, or both. We wanted to create an out-of-guest \ac{ACL}, which we call \ac{SACL}, for managing file access inside a VM. We call this mechanism Protecting Compromised Systems with a \ac{VMPCS-OGP}.

\par In our research, we developed a prototype for a file-access monitor and control outside a \ac{VM}. We used a 64-bit Ubuntu OS running on top of a Xen hypervisor. The prototype leverages the \ac{VMI} capability of the Xen hypervisor leveraged with the LibVMI \ac{API}~\cite{payne2011libvmi}, as well as DRAKVUF~\cite{lengyel2014drakvuf}, a system used for dynamic malware analysis. It includes a modified DRAKVUF implementation and prototypes of the \ac{ACL} kept on the hypervisor and enforced on the guest \ac{VM}. Our approach is to provide a more strict environment for file access.

\par In this work we tried to assess how we could leverage the introspection capabilities of the Xen hypervisor to improve the confidentiality, integrity, and availability mechanisms built into the \ac{OS}. Some of these cases included denying the root user access to parts of the file-system. We wanted to make a more fine-grained access control to fill the gap of the Linux native permission bits by denying file access to certain users that belong to a group with access. Furthermore, we wanted to alter the user permissions by keeping a \ac{SACL}. Moreover, as part of covering the tracks of the malicious activities, we wanted to try to enforce append-only permissions instead of write for specific cases of files, which include primarily log files, as we wanted to prevent a malicious action to be removed from any logs. 

\par This solution could potentially be used in a variety of platforms like \ac{IoT} or \ac{SCADA} systems, cellphones, cloud solutions; essentially, everything that runs on a virtualized environment. It could also be used to enhance the filesystem security of end-of-life systems that do not receive any security updates and are more susceptible to exploitation.

\section{Research Questions}\label{sec:question}
\par The primary issue we addressed in this research was whether we could enforce out-of-guest permissions to check access to the files of a system so that the attacker is not able to read or write critical files on the system. Following that we addressed:
\begin{itemize}
	\item What is the best way to implement a monitor for file access on the guest?
	\item What is the performance overhead?
	\item Can this mechanism be leveraged to identify a compromised system or a system actively being compromised?
	\item Is it manageable to monitor all files on a system or only specific ones?
	\item What is the best way to implement \ac{VMPCS-OGP} on a guest and still provide usability and protection?
	\item Can \ac{VMPCS-OGP} be used to discover how a system was compromised and the attackers' methods for compromising a system\footnote{sort of a honey pot approach.}?
	\item Can we return a valid error to the VM while denying access to a file so that it does not reveal the extra security check imposed by the hypervisor?
	\item Can we enforce an append-only write policy for files like logs?
\end{itemize}

\section{Organization}\label{sec:organization}
This paper is organized into five chapters. Chapter~\ref{ch:intro} introduces the concepts and thesis focus. Chapter~\ref{ch:background} covers some background information for the platform used in this research, as well as some of the security solutions already presented that make use of \ac{VMI}. Chapter~\ref{ch:chapter3} analyzes the design and methodology of the implemented mechanism; Chapter~\ref{ch:chapter4} discusses the performance testing results and presents our conclusions. Chapter~\ref{ch:chapter5} suggests possibilities for future work.




%
% (include other chapters here...)
%


% APPENDICES
% You have two recommended options for your appendix:
% a) A single appendix (with a single TOC entry)
% b) Multiple appendices. Look under the examples directory for a demo of
%   multiple appendices.
%
\NPSappendixTOC{[My Appendix Title]}
\label{app:A}

\definecolor{mGreen}{rgb}{0,0.6,0}
\definecolor{mGray}{rgb}{0.5,0.5,0.5}
\definecolor{mPurple}{rgb}{0.58,0,0.82}
\definecolor{backgroundColour}{rgb}{0.95,0.95,0.92}

\lstdefinestyle{CStyle}{
	%backgroundcolor=\color{backgroundColour},   
	commentstyle=\color{mGreen},
	keywordstyle=\color{blue},
	numberstyle=\tiny\color{mGray},
	stringstyle=\color{magenta},
	basicstyle=\scriptsize,
	breakatwhitespace=false,         
	breaklines=true,                 
	captionpos=b,                    
	keepspaces=true,                 
	%numbers=left,                    
	numbersep=5pt,                  
	showspaces=false,                
	showstringspaces=false,
	showtabs=false,                  
	tabsize=2,
	language=C
}

This appendix contains the script we created to test the various file access methods.
\fontfamily{qcr}\selectfont
\begin{lstlisting}[style=CStyle]
#include <stdio.h>
#include <stdlib.h>
#include <fcntl.h>
#include <unistd.h>

int main(int argc, char** argv){

  int fp = 0, r = 0;
  void * buf = malloc(64);
  char * a = "a";

  fp = open(argv[1], O_RDONLY);
  if (fp < 0){
    printf("Failure: Could not open file for reading. 
    Cannot copy.\n");
  } else {
    printf("Success: Opened file for reading. 
    Able to copy.\n");
    close(fp);
  }

  fp = open(argv[1], O_WRONLY);
  if (fp < 0){
    printf("Failure: Could not open file for writing.\n");
  } else {
    printf("Success: Opened file for writing.\n");
    close(fp);
  }

  fp = open(argv[2], O_RDWR | O_CREAT, 0666);
  if (fp < 0){
    printf("Failure: Could not create file.\n");
  } else {
    printf("Success: created file.\n");

    r = read(fp, buf, 1);
    if (r < 0) {
      printf("Failure: Could not write to file.\n");
    } else {
      printf("Success: Can read from file.\n");
    }

    r = write(fp, a, 1);
    if (r < 1) {
      printf("Failure: could not write to file.\n");
    } else {
      printf("Success: Can write to file.\n");
    }

    close(fp);
  }

  fp = unlink(argv[2]);
  if (fp != 0) {
    printf("Failure: Could not delete file.\n");
  } else {
    printf("Success: File deleted.\n");
  }

  fp = rename(argv[1], argv[3]);
  if (fp != 0) {
    printf("Failure: Could not move file.\n");
  } else {
    printf("Success: File moved.\n");
    fp = rename(argv[3], argv[1]);
  }

  return 0;
}

\end{lstlisting}
\fontfamily{ptm}\selectfont


% REFERENCES
% List all your BibTeX reference source files (ending in *.bib extension)
%
\NPSbibliography{thesis}


%
% This is the official end of the thesis.
%
\NPSend

% DISTRIBUTION LIST
% The list is automatically properly numbered
% and already populated with the mandatory recipients.
%
\NPSdistribution{Initial Distribution List}
\begin{distributionlist}
\item Defense Technical Information Center\\Ft. Belvoir, Virginia
\item Dudley Knox Library\\Naval Postgraduate School\\Monterey, California
%
%---- Other entries are no longer needed, because of Special Abstract Form
% Marine Corps students are required to show:
%\item Marine Corps Representative\\Naval Postgraduate School\\Monterey, California
%\item Directory, Training and Education, MCCDC, Code C46\\Quantico, Virginia
%\item Marine Corps Tactical System Support Activity (Attn: Operations Officer)\\Camp Pendleton, California
%
% Officer students in the Operations Research Program are also required to show:
%\item Director, Studies and Analysis Division, MCCDC, Code C45\\ Quantico, Virginia
%
% Officer students in the Space Ops/Space Engineering Program or in the Information Warfare/Information Systems and Operations are also required to show:
%\item Head, Information Operations and Space Integration Branch,\\ PLI/PP\&O/HQMC, Washington, DC
\end{distributionlist}


\end{document}

