% NOTE: Follow the capitalization from the "DoD Dictionary of Military
% and Associated Terms" (http://www.dtic.mil/doctrine/new_pubs/jp1_02.pdf)
% or from common use.

%% The NPS template readily supports two methods of generating list of acronyms.

%%
%% Option 1 - Powerful list of acronyms
%% ========

%% These acronyms are common and do not appear in our list of acronyms
%% We use a different way to define them, so they don't appear in the list:
\newacro{US}[U.S.]{United States}
\newacro{FBI}[FBI]{Federal Bureau of Investigations}
\newacro{CIA}[CIA]{Central Intelligence Agency}

% Mark the above as used, so they appear in short form by default:
\renewcommand{\NPSacrocommon}{
 \acused{US}
 \acused{FBI}
 \acused{CIA}
}
\NPSacrocommon{}

%% On the first line below, the longest acronym is placed in the 
%% square brackets.  This is used to size the column of the table 
%% correctly.
%% Other acronyms appear in the list of acronyms, in the order defined.
%%
\begin{acronym}[XXXXXXXXX] % longest acronym in these square brackets

%\acro{DoD}{Department of Defense}
%\acro{MASINT}{measurement and signature intelligence}
%\acro{NPS}{Naval Postgraduate School}
%\acro{SRWBR}{short range wide band radio}
%\acro{TCP}{Transmission Control Protocol}
%\acro{USN}{\ac{US} Navy}
%\acro{UDP}{User Datagram Protocol}
%\acro{USG}{United States government}
\acro{NIST}{National Institute of Standards and Technology}
\acro{OS}{Operating System}
\acro{VM}{Virtual Machines}
\acro{CPU}{Central Processing Unit}
\acro{VMI}{Virtual Machine Introspection}
\acro{ACL}{Access Control List}
\acro{SACL}{Shadow ACL}
\acro{VMPCS-OGP}{Virtual-Machine Protection and Checking System Using Out-Of-Guest Permissions}
\acro{API}{Application program interface}
\acro{VT}{Virtualization Technology}
\acro{PT}{Page Table}
\acro{EPT}{Extended Page Tables}
\acro{GMFN}{Guest Machine Frame Number}
\acro{MFN}{Machine Frame Number}
\acro{IOMMU}{Input/Output Memory Management Unit}
\acro{GVA}{Guest Virtual Address}
\acro{MAC}{Mandatory Access Control}
\acro{HAP}{High Assurance Processes}
\acro{OI}{Object Identifiers}
\acro{SGX}{Software Guard Extensions}
\acro{IDS}{Intrusion Detection System}
\acro{HIDS}{Host \ac{IDS}}
\acro{NIDS}{Network \ac{IDS}}
\acro{NIC}{Network Interface Card}
\acro{IoT}{Internet of Things}
\acro{SCADA}{Supervisory control and data acquisition}
\acro{PAM}{pluggable authentication module}


\end{acronym}

%% Using documentclass[acronym,...]
%% --------------------------------
%% If you include the 'acronym' package in the documentclass[] options
%% of the NPS template, it will only print items in this acro list that are
%% actually used in your paper.  In this manner, your acronym list will always
%% be up to date.  You can then reuse this acronyms.tex in other documents
%% as you continue to expand your acronyms in use.
%%
%% If you don't include the 'acronym' package explicitly, the package is
%% auto-included by the NPS Thesis Template, but all the acro{} entries
%% will be put into your thesis, whether you actually use them or not.
%%
%% Using the macros:
%% ----------------
%% It will track the usage of acronyms and uses the long form on their
%% first use.  This ensures consistency throughout your document even
%% in different revisions of your document. Your writing is also more portable.
%%
%% To use your acronym in a paragraph use \ac{shortname} or \acplural{shortname}
%% \acp{shortname} for plural shorthand.
%% Ex:  The \ac{FFT2} is performed on the data. 
%% The \acplural{FFT2} are efficient.
%% This command is how LaTeX tracks their usage.
%%
%% Using documentclass[acronym,index,...]
%% --------------------------------------
%% You may want to use this:
%%   \newcommand{\FFT2}{\ac{FFT2}\index{fast Fourier Transform}\xspace}
%% to save some time and typing.  Then you can do:
%% Ex: The \FFT2 is performed on the data.
%% The acronym will be used correctly, and the index will contain the 
%% use of the keyword. 
%%
%% The command's \xspace ensures that the spacing after the word is
%% handled correctly (ie, last word of a sentence gets a period on 
%% the word).


%%
%% Option 2 - Build your own list of acronyms
%% =========
%% The spacing with this option is wonky. It is no longer supported or reviewed.

%% With this option, you are responsible to write out the long form on the
%% first use, then you can simply use the acronym in your paragraphs that
%% follow.  This can become cumbersome during revisions to your thesis
%% because it is your responsibility to know if the acronym has been used yet.

%\begin{description}
%\item[MASINT] measurement and signature intelligence
%\item[NPS] Naval Postgraduate School
%\item[SRWBR] short range wide band radio
%\item[TCP] Transmission Control Protocol
%\item[UDP] User Datagram Protocol
%\item[USG] United States government
%\end{description}

