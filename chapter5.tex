
\chapter{Conclusion and Future Work}\label{ch:chapter5}

\par In this chapter, we present the benefits of \codeft{ferify} and the lessons we learned during its development. We also present some limitations of a security solution that is completely transparent to the guest \ac{VM}, as we encountered them. Finally, we present some ideas for future development of \codeft{ferify}.

\section{Conclusion}

\par In this thesis we developed \codeft{ferify}, a virtual file-protection system for preventing a range of \emph{zero-day} attacks. As mentioned in Chapter \ref{ch:background}, other virtualization solutions exist, with others being continuously developed. We believe that \codeft{ferify} has some unique characteristics which make it more efficient than the already-existing solutions. As we observed them, the benefits of \codeft{ferify}, as well as some limitations of a security solution that is completely transparent to the guest \ac{VM}, can be summarized as follows:

\begin{itemize}
\item We decided to use DRAKVUF~\cite{lengyel2014drakvuf} as our base platform. This inherently gives to \codeft{ferify} a \emph{zero-footprint} in the guest \ac{VM}, making it a full \emph{out-\ac{VM}} solution which \emph{cannot be detected} by the guest \ac{VM}. Although there are techniques for detecting if an \ac{OS} is virtualized or not, there is no way to detect if \codeft{ferify} is running, nor how to bypass the settings of the \ac{SACL}.

\item Making \codeft{ferify} a complete out-\ac{VM} solution in which all code and information is stored on the hypervisor makes interacting and subverting \codeft{ferify} almost impossible, giving our solution a great isolation from the guest \ac{OS}. Hypervisor-based attacks, are outside the scope of this thesis: we expect that hypervisors will become more secure and robust over time.

\item Using \codeft{ferify} can have a significant improvement on an \ac{OS}'s file \emph{confidentiality}, \emph{integrity}, and \emph{availability}. As we have managed to tighten the Linux file permissions, we can enforce a different \emph{user-based}  \ac{ACL} policy than the one in the guest \ac{OS}, protecting critical files and information, in a transparent way. We can deny reading, modification, and deletion of files, as well as execution of programs, on a \emph{white-list} basis. Applying all these for the \emph{root} account also is a significant improvement in information security and system integrity, as until now the system's administrator could access everything stored on the \ac{OS}.

\item \codeft{ferify} also provides \emph{kernel security} and \emph{integrity} by denying kernel module loading and new kernel booting. This step ensures that the guest \ac{OS} kernel will remain \emph{unmodified} by attackers, and that existing and new \emph{zero-day} attacks will not work on a \codeft{ferify} protected \ac{VM}.

\item LibVMI currently provides an \ac{API} for accessing and modifying a \ac{VM}'s \ac{CPU} registers or memory contents. It does not provide functions to access files and devices. This is an important limitation on what introspection can achieve. For cases like \codeft{ferify}, protecting files must be done on a different level than that of the actual file. This is the reason we had to trap the system calls and introduce an insignificant performance overhead.

\item The performance overhead, as measured in Section~\ref{sec:performance}, is a significant factor that needs to be considered. Trapping of many system calls can result in a less-usable environment. Our measurements were performed using only one \ac{VM}. In a more expanded environment, where many \acp{VM} are running, the hypervisor-\ac{VM} switch overhead can increase significantly and be a limiting factor in the use of virtualization security applications.

\item In Chapter~\ref{ch:background} we mentioned the data semantic gap that exists between the hypervisor and a \ac{VM}. Making assumptions and inferences of the high-level actions in a \ac{VM} by examining its internal state is at best extremely difficult, if not impossible. To achieve that, we need to know the internals of the guest \ac{OS} at a deep level in order to be able to retrieve information that allows us to recreate the high-level actions. This data semantic gap is a huge obstacle and limitation of an out-\ac{VM} security solution. In the case of \codeft{ferify}, the design was such that we did not need to make any inferences about the state of the guest \ac{OS}. The file access control policy monitoring does not need to implement intricate relations between different kernel memory structures or user-space memory. This allowed for a simpler design, but could become more complex as new features might need access to additional data.
 
\end{itemize} 


\section{Lessons Learned}

\par During the development of \codeft{ferify} we encountered some problems, which we were required to solve in order to continue our progress.  

\par Creating an out-\ac{VM} solution has some challenges, with the most important being the data semantic gap mentioned in Chapter~\ref{ch:background}. Although for \codeft{ferify}, there is need for minimal context of what is happening in the guest \ac{VM}, we needed to bridge that gap in some cases. This bridging required deep research of the Linux kernel to determine where the required information is stored and how to recreate some information that is not directly stored in the kernel. In the case of the development of a solution that is required to make some decisions and inferences based on higher level information, bridging this semantic gap could be extremely challenging.

\par To make a more complete and robust application, building a security solution is a complex process which requires several iterations and revisiting of already-implemented parts for improvements and integration of newer features. During this thesis, we initially considered trapping only \emph{three} system calls. As \codeft{ferify} evolved, and we discovered other vectors of file access, we finally trapped a total of \emph{nineteen} system calls so that we could monitor \emph{all} possible ways of accessing a file.


\section{Future Work}

\par As in every case of a developed application, \codeft{ferify} can also be improved to provide improved efficiency and better monitoring. Although we believe that we have addressed \emph{all} possible ways a file can be accessed, there are ways \codeft{ferify} can be improved:

\begin{itemize}

\item Finding the thread count of a process and monitoring their additions and exits should be added in \codeft{ferify}'s logic. Although the basic functionality is present, missing the initial thread count of a process does not allow us to know when a process completely terminates, and is not always removed from our task-list. This can result in some errors during the validation of the process's identity, in the case of the creation of a new process with a duplicate \codeft{pid} as the one we did not detect exiting. Although we have trapped the \codeft{exit\_group()} system call, we cannot be sure that the last thread will invoke it and not the \codeft{exit()} system call.
Furthermore, more research and testing is required to verify whether or not multi-threaded processes have the same \codeft{CR3} value between threads, and keep track of it for use in the process identity validation.


\item In our test environment we used a single-\ac{CPU} \ac{VM}. Future work might be needed to modify \codeft{ferify} to work on multi-\ac{CPU} \acp{VM}. DRAKVUF, when running a callback function, provides the information about the virtual-\ac{CPU} that causes the \ac{VM}-exit event. That information can be used to adjust \codeft{ferify}'s code, if needed, to trap and handle \emph{all} system calls from \emph{all} virtual-\acp{CPU} of the guest \ac{VM}.

\item Reloading the \acp{SACL} while \codeft{ferify} is running is a desired feature that would also make \codeft{ferify} more robust and secure. This could be done by registering a new signal handler, or modifying an existing one, to destroy and recreate the hash-tables from the new \acp{SACL}. Additionally, the development of an administrative interface that allows easier management of the \acp{SACL} will increase the usability of the system. Currently, if there is a need to change the \acp{SACL}, we need to terminate \codeft{ferify} and relaunch it, for the changes to take effect, leaving a window of vulnerability between the termination and re-initialization of \codeft{ferify}. 

\item \codeft{ferify} currently blocks all kernel modules that users try to load, although there is room for improvement. Usually, after the system has booted there is no need to load a new kernel module. There are cases though, that a \emph{trusted} kernel module needs to be loaded. Allowing digitally signed modules can improve usability of the system, as they can be considered trusted and allowed to execute. This could be implemented by reading the digital signature from the hypervisor and then checking its validity before continuing execution of the guest \ac{VM}.

\item Currently, \codeft{ferify} provides logging of important events in a \ac{VM}. A notification mechanism to alert the system's administrators of illegal access would improve incident response-time. This mechanism could potentially include options like auto-pausing, for handling a maliciously-accessed \ac{VM}.

\item Finally, as our development only used the Ubuntu distribution, additional testing must be done to ensure compatibility with other Linux distribution. We expect that this will not be an issue as most Linux distributions use the same kernel.


\end{itemize}

\par The above list is by no means exhaustive. As new problems or desired features surface, other features or enhancements are likely to be required. 


%\begin{itemize}
%	\item Multiprocess applications? Seems like I get sometimes different process base.
%	\item Multithreadded applications. 
%	\item Implement on the fly \ac{SACL} updates. 
%	\item Modify code as necessary to work in multi-CPU \acp{VM}. Maybe it works?
%\end{itemize}

