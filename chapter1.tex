\chapter{Introduction}\label{ch:intro}

System virtualization, which has been increasing in popularity over the last years, makes it possible to run many and different \acp{OS} on the same physical machine. \acp{VM}, are run independently of each other on the same physical machine, known as a host, without any indication that there is another \ac{OS} running on the same host. The software that facilitates this resource sharing capability is called a hypervisor.

\par The emergence of cloud computing has increased the need for many new and different services from different global vendors. According to the \ac{NIST}~\cite{mell2011nist}, "cloud computing is a model for enabling ubiquitous, convenient, on-demand network access to a shared pool of configurable computing resources (e.g., networks, servers, storage, applications, and services) that can be rapidly provisioned and released with minimal management effort or service provider interaction".  

\par Virtualization solved the increasing requirement of resources for these services to run on. Instead of having many separate physical machines running the required different software, usually resulting in underutilization, one machine, with better technical specifications and capabilities like more memory capacity and multi-core \acp{CPU}, was used, and with virtualization, each vendor could run his services on a dedicated \ac{VM}. 

\par In order to improve network security on these machines, as well as redundancy among different services, instead of having one VM running all the services required by one vendor, service providers started using many different \ac{VM}s, each requiring fewer resources and running one or just a couple of services. This way, if one VM fails, the rest of the services keep running. Furthermore, having each \ac{VM} run only a few services significantly reduces the attack surface available for possible vulnerability exploitation.

\par This increase in the use of virtualization has driven hardware manufacturers, like Intel and AMD, to introduce special virtualization \ac{CPU} instructions, to facilitate better, more reliable and secure allocation, sharing, usage and performance of \ac{VM}s. 

\section{Problem Statement}\label{sec:problem} 
When an \ac{OS} runs directly on a physical machine, it allocates and uses its resources to protect itself from network or other types of attacks. But, when it runs on a virtualization platform, the hypervisor stands between the hardware and the running software, and can see what is happening inside a \ac{VM}. 

\par Despite the evolution of \ac{CPU} virtualization instructions, and the continuous development of more efficient and secure hypervisors, the bottom-line remains the same. A \ac{VM} is still a system, with all the vulnerabilities of its running \ac{OS} and software, and at some point in time, it will be the victim of a successful exploitation. 

\par The native Linux file permission system, although simple to manage and efficient, lacks fine-grained user/group access to files. Once users belong to a group, nothing prohibits them from accessing all the files accessible to that group. Furthermore, when attackers gains access to a system, they will usually try to escalate their privileges by having access to the root account. From that point there is nothing out of reach; the attacker has unrestricted access to the entire system and is free to read and modify files and change the system's configuration to his liking, to serve his purposes.

\par Garfinkel et al~\cite{garfinkel2003virtual} introduces a new technique which leverages the hypervisor's viewing ability. \ac{VMI} is the “approach of inspecting a virtual machine from the outside for the purpose of analyzing the software running inside it.” Outside in this context means that the inspecting application resides outside the monitored \ac{VM} and can access the \ac{VM}'s state through the hypervisor. Because a system will be eventually subverted, we want to leverage the introspection capability of a hypervisor to try to protect critical files for the OS, the user, or both. We want to create an out-of-guest \ac{ACL}, which we call \ac{SACL}, for managing file access inside a VM. We call this mechanism Protecting Compromised Systems with a \ac{VMPCS-OGP}.

\par In our research, we developed a prototype for a file access monitor and control outside a \ac{VM}. We used a 64-bit Ubuntu OS running on top of a Xen hypervisor. The prototype leverages the \ac{VMI} capability of the Xen hypervisor leveraged with the LibVMI \ac{API}~\cite{payne2011libvmi}, as well as DRAKVUF~\cite{lengyel2014drakvuf}, a system used for dynamic malware analysis. It includes a modified DRAKVUF implementation, as well as prototypes of the \ac{ACL} kept on the hypervisor, to be enforced on the guest \ac{VM}. Our approach is to provide a more strict environment for file access.

\par In this work we tried to assess how we can leverage the introspection capabilities of the Xen hypervisor to improve the confidentiality, integrity and availability mechanisms built into the \ac{OS}. Some of these cases include denying access to the root user, who has access to the entire filesystem. We want to make a more fine-grained access control to fill the gap of the Linux native permission bits by denying access to files on users that belong to a group with access. Furthermore, we want to alter the user permissions by keeping a \ac{SACL}. Moreover, we will try to enforce append-only permission instead of generic write for specific cases of files, which include primarily log files, as we would like to prevent a malicious action to be removed from any logs, as part of covering the tracks of the malicious activity. 

\par This solution can potentially be used in a variety of platforms like \ac{IoT} or \ac{SCADA} systems, cellphones, cloud solutions, essentially everything that runs on a virtualized environment. It could also be used to enhance the filesystem security of end-of-life systems that do not receive any security updates and are more susceptible to exploitation.

\section{Research Questions}\label{sec:question}
\par The primary issue we addressed in this research is whether we can enforce out-of-guest permissions to check access to the files of a system so that the attacker is not able to read or write critical files on the system. Following that we will address:
\begin{itemize}
	\item What is the best way to implement a monitor for file access on the guest.
	\item What is the performance overhead.
	\item If this mechanism can be leveraged to identify a compromised system or a system actively being compromised.
	\item If it is manageable to monitor all files on a system or only specific ones. 
	\item What is the best way to implement \ac{VMPCS-OGP} on a guest and still provide usability and protection.
	\item If \ac{VMPCS-OGP} can be used to discover how a system was compromised and attacker methods in compromising a system – sort of a honey pot approach.
	\item If we can return a valid error to the VM while denying access to a file, so that it does not reveal the extra security check imposed by the hypervisor.
	\item Can we enforce an append-only write policy for files like logs.
\end{itemize}

\section{Organization}\label{sec:organization}
This paper is organized into five chapters. Chapter~\ref{ch:intro} introduces the concepts and thesis focus. Chapter~\ref{ch:background} covers some background information for the platform used in this research, as well as some of the security solution already presented that make use of \ac{VMI}. Chapter~\ref{ch:chapter3} analyzes the design and methodology of the implemented mechanism, and Chapter~\ref{ch:chapter4} discusses the performance testing results and presents our conclusions. Chapter~\ref{ch:chapter5} suggests future work.



