\chapter{Introduction}\label{ch:intro}

System virtualization, which has been increasing in popularity over the last few years, makes it possible to run multiple different \acp{OS} on the same physical machine. \acp{VM} are run independently of each other on the same physical machine, known as a host, without any indication that there is another \ac{OS} running on the same host. The software that facilitates this resource sharing capability is called a hypervisor.

\par The emergence of cloud computing has increased the need for many new and different services from global vendors. According to the \ac{NIST}~\cite{mell2011nist}, "Cloud computing is a model for enabling ubiquitous, convenient, on-demand network access to a shared pool of configurable computing resources (e.g., networks, servers, storage, applications, and services) that can be rapidly provisioned and released with minimal management effort or service provider interaction."  

\par Virtualization solved the increasing requirement for resources for these services to run on. Instead of having many separate physical machines running the required various software, which usually results in under-utilization, one machine with better technical specifications and capabilities\footnote{such as more memory capacity and multi-core \acp{CPU}} was used; with virtualization, each vendor could run services on a dedicated \ac{VM}. 

\par In order to improve network security on these machines, as well as redundancy among different services service providers started using many different \acp{VM} per vendor, instead of having one \ac{VM} running all the required services. In addition to requiring fewer resources by running one or two services, vendors could be reassured by knowing that if one VM fails, the rest of the services keep running. Furthermore, having each \ac{VM} run only a few services significantly reduces the attack surface available for possible vulnerability exploitation.

\par This increase in the use of virtualization has driven hardware manufacturers, like Intel and AMD, to introduce special virtualization \ac{CPU} instructions that demonstrate better, more reliable, and more secure allocation, sharing, usage, and performance.

\section{Problem Statement}\label{sec:problem} 
When an \ac{OS} runs directly on a physical machine, it allocates and uses its resources to protect itself from network or other types of attacks. When it runs on a virtualization platform however, the hypervisor stands between the hardware and the running software, and can monitor what is happening inside a \ac{VM}. 

\par Despite the evolution of \ac{CPU} virtualization instructions and the continuous development of more efficient and secure hypervisors, the bottom-line remains the same: a \ac{VM} is still a system with all the vulnerabilities of its running \ac{OS} and software. At some point in time, it will be the victim of a successful exploitation. 

\par Although simple to manage and efficient, the native Linux file permission system lacks fine-grained user/group access to files. Once users belong to a group, nothing prohibits them from accessing all the files accessible to that group. Furthermore, when attackers gain access to a system, they will usually try to escalate their privileges by having access to the root account. After privilege escalation, there is unrestricted access to the entire system and nothing out of reach; the attackers are free to read and modify files and change the system's configuration to their liking in order to serve their purposes.

\par Garfinkel et al.~\cite{garfinkel2003virtual} introduced a new technique which leverages the hypervisor's viewing ability. \ac{VMI} is the “approach of inspecting a virtual machine from the outside for the purpose of analyzing the software running inside it.” In this context, outside means that the inspecting application resides outside the monitored \ac{VM} and can access the \ac{VM}'s state through the hypervisor. Because a system will be eventually subverted, we wanted to leverage the introspection capability of a hypervisor to try to protect critical files for the OS, the user, or both. We wanted to create an out-of-guest \ac{ACL}, which we call \ac{SACL}, for managing file access inside a VM. We call this mechanism Protecting Compromised Systems with a \ac{VMPCS-OGP}.

\par In our research, we developed a prototype for a file-access monitor and control outside a \ac{VM}. We used a 64-bit Ubuntu OS running on top of a Xen hypervisor. The prototype leverages the \ac{VMI} capability of the Xen hypervisor leveraged with the LibVMI \ac{API}~\cite{payne2011libvmi}, as well as DRAKVUF~\cite{lengyel2014drakvuf}, a system used for dynamic malware analysis. It includes a modified DRAKVUF implementation and prototypes of the \ac{ACL} kept on the hypervisor and enforced on the guest \ac{VM}. Our approach is to provide a more strict environment for file access.

\par In this work we tried to assess how we could leverage the introspection capabilities of the Xen hypervisor to improve the confidentiality, integrity, and availability mechanisms built into the \ac{OS}. Some of these cases included denying the root user access to parts of the file-system. We wanted to make a more fine-grained access control to fill the gap of the Linux native permission bits by denying file access to certain users that belong to a group with access. Furthermore, we wanted to alter the user permissions by keeping a \ac{SACL}. Moreover, as part of covering the tracks of the malicious activities, we wanted to try to enforce append-only permissions instead of write for specific cases of files, which include primarily log files, as we wanted to prevent a malicious action to be removed from any logs. 

\par This solution could potentially be used in a variety of platforms like \ac{IoT} or \ac{SCADA} systems, cellphones, cloud solutions; essentially, everything that runs on a virtualized environment. It could also be used to enhance the filesystem security of end-of-life systems that do not receive any security updates and are more susceptible to exploitation.

\section{Research Questions}\label{sec:question}
\par The primary issue we addressed in this research was whether we could enforce out-of-guest permissions to check access to the files of a system so that the attacker is not able to read or write critical files on the system. Following that we addressed:
\begin{itemize}
	\item What is the best way to implement a monitor for file access on the guest?
	\item What is the performance overhead?
	\item Can this mechanism be leveraged to identify a compromised system or a system actively being compromised?
	\item Is it manageable to monitor all files on a system or only specific ones?
	\item What is the best way to implement \ac{VMPCS-OGP} on a guest and still provide usability and protection?
	\item Can \ac{VMPCS-OGP} be used to discover how a system was compromised and the attackers' methods for compromising a system\footnote{sort of a honey pot approach.}?
	\item Can we return a valid error to the VM while denying access to a file so that it does not reveal the extra security check imposed by the hypervisor?
	\item Can we enforce an append-only write policy for files like logs?
\end{itemize}

\section{Organization}\label{sec:organization}
This paper is organized into five chapters. Chapter~\ref{ch:intro} introduces the concepts and thesis focus. Chapter~\ref{ch:background} covers some background information for the platform used in this research, as well as some of the security solutions already presented that make use of \ac{VMI}. Chapter~\ref{ch:chapter3} analyzes the design and methodology of the implemented mechanism; Chapter~\ref{ch:chapter4} discusses the performance testing results and presents our conclusions. Chapter~\ref{ch:chapter5} suggests possibilities for future work.



