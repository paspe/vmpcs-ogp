\chapter{Introduction}\label{ch:intro}

System virtualization has been increasing in popularity over the last years. It makes it possible to run many and different Operating Systems (OS) on the same physical machine. That kind of \ac{OS}, known as \ac{VM}, is run independently of each other on the same physical machine, known as host, without any indication that there is another \ac{OS} running on the same Host. It is essentially a resource sharing mechanism. The software that facilitates this capability is called a hypervisor.
\par The emergence of cloud computing has increased the requirement of many new and different services from different vendors, usually around the globe. According to the \ac{NIST}~\cite{mell2011nist}, "cloud computing is a model for enabling ubiquitous, convenient, on-demand network access to a shared pool of configurable computing resources (e.g., networks, servers, storage, applications, and services) that can be rapidly provisioned and released with minimal management effort or service provider interaction".  
\par Virtualization gave a solution to the increasing requirement of resources for these services to run on. Instead of having many separate physical machines running the required different software, usually resulting in underutilization, one machine, with better specifications, was used, and with the use of virtualization, each vendor could run his services on a dedicated \ac{VM}. 
\par Also, in order to improve network security on these machines, as well as redundancy among different services, instead of having one VM running all the services required by one vendor, service providers started using many different \ac{VM}s, each requiring less resources and running one or just a couple of services. This way, if one VM fails, the rest of the services keep running. Furthermore, by having each \ac{VM} run only a few services, the attack surface available for possible vulnerability exploitation is reduced significantly.
\par This increase in the use of virtualization has driven even hardware manufacturers, like Intel and AMD, to introduce special virtualization \ac{CPU} instructions, to facilitate better, more reliable and secure allocation, sharing, usage and performance of \ac{VM}s. 
\par Despite the evolution of \ac{CPU} virtualization instructions, and the continuous development of more efficient and secure hypervisors, the bottom-line remains the same. A \ac{VM} is still a system, with all the vulnerabilities of its running \ac{OS} and software, which at some point in time, will be the victim of a successful exploitation. 

\section{Problem Statement}\label{sec:problem} The motivation for this research came from that idea exactly. When an \ac{OS} is running directly on a physical machine, it is on its own to allocate and use its resources and protect itself from network or other types of attacks. But, when it runs on a virtualization platform, the hypervisor stands between the hardware and the running software, and has full visibility in what is happening inside a \ac{VM}. 

\par The native Linux file permission system, although simple to manage and efficient, it lacks fine-grained user/group access to files. Once a user belongs to a group, nothing prohibits him from accessing all the files accessible to that group. Furthermore, when an attacker gains access to a system, he usually will try to escalate his privileges by having access to the root account. From that point there is nothing out of reach and the attacker has unrestricted access to the entire system and is free to read and modify files and change the system's configuration to his liking, to serve his purposes.

\par In~\cite{garfinkel2003virtual}, the author names a new technique which leverages this viewing ability of the hypervisor. \ac{VMI} is the “approach of inspecting a virtual machine from the outside for the purpose of analyzing the software running inside it”. Having in mind that a system will be eventually subverted, we want to leverage the introspection capability of a hypervisor to try to protect critical files for the OS, the user or both. We want to create an gut-of-guest \ac{ACL}, which we call \ac{SACL}, for managing file access inside a VM. We call this mechanism Protecting Compromised Systems with a \ac{VMPCS-OGP}.

\par In our research, we develop a prototype for file access monitor and control outside a \ac{VM}. We use a 64-bit Ubuntu OS running on top of a Xen hypervisor. The prototype leverages the \ac{VMI} capability of the Xen hypervisor leveraged with the LibVMI \ac{API}~\cite{payne2011libvmi}, as well as DRAKVUF~\cite{lengyel2014drakvuf}, a system used for dynamic malware analysis. It includes a modified DRAKVUF implementation, as well as prototypes of the \ac{ACL}, kept on the hypervisor, to be enforced on the guest \ac{VM}. Our approach is to provide a more tightened but fine-grained environment.

\par In this work we will try to assess how we can leverage the introspection capabilities of the Xen hypervisor to improve the \ac{OS} built-in confidentiality and integrity mechanisms. Some of these cases include denying access to the root user, who has access to the entire filesystem. We want to make a more fine-grained access control to fill the gap of the Linux native permission bits, by denying access to files on users that belong to a group with access. Furthermore, we want to alter the user permissions, by keeping a \ac{SACL}. 

\par This solution can potentially be used in a variety of platforms like \ac{IoT} or \ac{SCADA} systems, cellphones, cloud solutions, essentially everything that runs on a virtualized environment.

\section{Research Questions}\label{sec:question}
\par The primary issue we will address in this research is whether we can enforce out-of-guest permissions to check access to the files of a system, so that the attacker is not able to read or write critical files on the system. Following that we will address:
\begin{itemize}
	\item What is the best way to implement a monitor for file access on the guest.
	\item What is the performance overhead.
	\item If this mechanism can be leveraged to identify a compromised system or a system actively being compromised.
	\item If it is manageable to monitor all files on a system or only specific ones. 
	\item What is the best way to implement \ac{VMPCS-OGP} on a guest and still provide usability and protection.
	\item If \ac{VMPCS-OGP} can be used to discover how a system was compromised and attacker methods in compromising a system – sort of honey pot approach.
	\item If we can return a valid error to the VM while denying access to a file, so that it does not reveal the extra security check imposed by the hypervisor.
\end{itemize}

\section{Organization}\label{sec:organization}
This paper is organized into five chapters. Chapter~\ref{ch:intro} introduces the concepts and thesis focus. Chapter~\ref{ch:background} covers some background information for this thesis research used platform, as well as some of the security solution already presented, that make use of VMI. Chapter~\ref{ch:chapter3} analyzes the design and methodology of the implemented mechanism, and Chapter~\ref{ch:chapter4} discusses the performance testing results and presents our conclusions. Chapter~\ref{ch:chapter5} suggests future work.



